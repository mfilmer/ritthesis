% This class inherits from the LaTeX standard report class and should support
% all of the options supported by it, as well as a few extra. These extra
% options are described here.
% Options to adjust line spacing:
% singlespacing, onehalfspacing, doublespacing
% Only one should be used at a time, and the default is doublespacing
% By default the three main lists (toc, lof, lot) are single spaced and the list
% of symbols is 1.5 line spacing. To return these four lists to double spacing
% use the option:
% doublespacelists
% If the microsystems title page and signature sheet should be used:
% microsys
% When using this option, only the \specialcommittee, and \certifiedby macros
% should be used. \certifiedby has the same syntax as \specialcommittee and 
% should include the list of names to appear under "Certified By" on the 
% signature sheet. \advisor and \committee should not be used as these don't 
% provide all the required options for the microsystems signature sheet.
% Additionally, the \setadvisor macro should be used (in addition to a
% \specialcommittee for the advisor) to store the advisor's name for use on the
% title page
\documentclass{ritthesis}

%%%%%%%%%%%%%%%%%%%%
%%%%% Packages %%%%%
%%%%%%%%%%%%%%%%%%%%
% Used for filler text but will almost certainly be of no use in a real thesis
\usepackage{lipsum}
% Generally you will want to use the graphicx package to display images
\usepackage{graphicx}
% You can set the graphicspath if you want to use pictures no in the same
% directory as the .tex file
%\graphicspath{{images/}}

% If subfigures or subtables are desires, the subcaption package is the
% recommended way to go. This package should not be used unless subfigures or
% subtables (or any other functionality it provides) is desired. It will not
% work if the subfig package is used.
\usepackage{subcaption}

% This command is used to set a consistent head height. If there are Fancyhdr
% warnings that it is too small, increase it. This should not happen unless
% headers of more than two lines are used. If only single line headers are used,
% this line can likely be removed
%\setlength{\headheight}{23pt}

% Use these commands to set up the author, title, and date
\author{Matthew J. Filmer}
% You most likely need to insert a line break or two to make the title wrap
% more evenly
\title{A \LaTeX{} Document Class for a Masters Thesis\\in Microelectronic Engineering at RIT}
\date{June 23, 2014}

% You can further customize the apperance of the title and author by
% uncommenting and changing:
% \makeatletter
% \renewcommand{\displaytitle}{\textbf{\large\@title}}
% \renewcommand{\displayauthor}{\textsc{\@author}}
% \makeatother
% But I like small caps, so that is how I made it

% There are also a bunch of other commands that can be changed, were this not
% a masters thesis in microelectronic engineering at RIT. Those would be the
% following:
% \renewcommand{\universityname}{Rochester Institute of Technology}
% \renewcommand{\collegename}{Kate Gleason College of Engineering}
% \renewcommand{\departmentname}{Department of Electrical and Microelectronic Engineering}
% \renewcommand{\logo}{kgcoelogohoriz}
% \renewcommand{\degreename}{Master of Science}
% \renewcommand{\degreefield}{Microelectronic Engineering}

% Each of these commands can be repeated as many times as necessary
% Used to create the lines on the signature sheet. There is an optional first
% argument for each of these to specify title (the default is Dr.)
\advisor{Assistant Professor}{Joe}{Smith}
\committee{Dean of the College}{Philip}{Jones}
\addcommittee[Mr.]{Bystander}{Robert}{van Rosenson XXXVIII}{King of the Committee}

% These are only used for the microsys signature sheet
\certifiedby[Mr.]{Bystander}{Robert}{van Rosenson XXXVIII}{King of the Committee}
\certifiedby[Mr.]{Bystander}{Robert}{van Rosenson XXXVIII}{King of the Committee}

% You can customize the length of the signature line by uncommenting and
% changing the following line. The default is 25em, and is close to the maximum
% line length
%\renewcommand{\siglinelength}{25em}

% This command makes testing the proposal option easier
% it should not be used in a real document
% It just redefines \chapter to be the same as \section
\providecommand{\chapter}[1]{\section{#1}}

% Begin the document, the way you always do in LaTeX
\begin{document}

% This starts the pages that get roman numerals at the beginning of the
% document. Generally is the first thing in the document. 
\makefrontmatter

% Instead of \makefrontmatter it is possible to do what thath macro does
% manually, if more control was desired. If that is the case, uncomment and
% change, or just use the individual macros:
% \renewcommand{\makefrontmatter}{\maketitlepage\makesignaturepage\makecopyrightpage}
% Each of these commands can also be changed, but for details on that you
% should look in ritthesis.cls to see how they are defined

% The acknowledgments environment will create an acknowledgments section,
% complete with a correctly spelled title. Change the following to rename
%\renewcommand{\acknowledgmentsname}{Acknowledgments}
\begin{acknowledgments}
\lipsum[4-6]
\end{acknowledgments}

% The dedication environment will create a dedication section, change the
% following to rename
%\renewcommand{\dedicationname}{Dedication}
\begin{dedication}
\lipsum[7]
\end{dedication}

% The abstract environment is very similar to the acknowledgments environment
% and is used the same. Change the following to rename
%\renewcommand{\abstractname}{Abstract}
\begin{abstract}
\lipsum[1-2] 
\end{abstract}

% Makes the table of contents, list of tables, and list of figures, in that
% order. This command makes a table of contents entry for each of these lists
\makealllists

% Here you can generate a list of symbols, or a list of abbreviations, or whatever
% other custom lists you may desire. These lists are built using the longtable
% environment, and takes similar arguments
% You can change the list of symbols name by either uncommenting the following line
% and making the desired change, or by providing the optional argument to the
% listofsymbols environment
%\renewcommand{\listofsymbname}{List of Symbols}
% The first argument is column alignment options and uses the same syntax as the
% argument for the longtable environment (which is similar to the tabular environment)
% The second argument is the column headers, separated by ailgnment characters (&)
% If you want multiple extra tables, such as a table of acronyms, and a table of symbols
% you can just use this environment multiple times, with different names
\begin{listofsymbols}[List of Symbols]{lp{0.6\linewidth}r}{Term & Description & Units/Value}
$C_D^\prime$ 	& Depletion region capacitance per unit area 			& F/cm\textsuperscript{2}\\
$C_{ox}^\prime$ & Oxide capacitance per unit area 						& F/cm\textsuperscript{2}\\
$\mathcal{E}$	& Electric field 										& V/cm\\
$E_c$ 			& Energy at the conduction band edge 					& eV\\
$E_F$			& Fermi level											& eV\\
$E_g$			& Band gap energy 										& eV\\
$E_v$ 			& Energy at the valence band edge 						& eV\\
$E_x, E_y$		& Electric field normal and parallel to the gate		& V/cm\\
$\hbar$			& Reduced Planck constant 								& $6.582\times 10^{-16}$ eV$\cdot$s\\
$I_D$			& Drain current											& A\\
$J_t$			& Band to band tunneling current density 				& A/cm\textsuperscript{2}\\
$k$				& Boltzmann's constant 									& $8.617\times 10^{-5}$ eV/K\\
$m^*$			& Carrier effective mass 								& kg\\
$N_A$			& Acceptor concentration 								& cm\textsuperscript{-3}\\
$N_c$			& Effective density of states in the conduction band 	& cm\textsuperscript{-3}\\
$n_i$ 			& Intrinsic carrier concentration						& cm\textsuperscript{-3}\\
$q$				& Elementary charge  									& $1.602\times 10^{-19}$ C\\
$S$ 			& Subthreshold swing 									& V/dec\\
$T$				& Temperature											& K\\
$V_{DS}$		& Drain--Source voltage 								& V\\
$V_{eff}$		& Effective reverse bias voltage						& V\\
$V_{FB}$ 		& Flatband voltage 										& V\\
$V_G$ 			& Gate voltage 											& V\\
$\epsilon_s$	& Permittivity of a semiconductor 						& F/cm\\
$\mu_n, \mu_p$  & Mobility of electrons and holes respectively 			& cm/V$\cdot$s\\
$\Psi_s$ 		& Surface potential										& V\\
\end{listofsymbols}

% The \mainmatter macro starts the main matter. This starts arabic numbering,
% and adds a header (except on the first page of a chapter). 
\mainmatter

% This class uses the fancyhdr package to build the headers. You could probably
% customize the headers by changing the plain page style, as well as the fancy
% page style. There is no guarantee anything you do works though. (In fact,
% there isn't even a guarantee anything in ritthesis.cls works either)

% Likewise, the chapter and section headers are formatted using the titlesec
% package. You should be able to customize them by using that package's
% commands, specifically \titleformat

% Begin new chapters the way you always do it LaTeX
\chapter{Doing things the hard way}

% Same goes for sections
\section{First sections aren't as much fun as first comments}
The first paragraph in a chapter\footnote{and probably section} is not indented. I can probably change this, but it is the \LaTeX{} default. Here is an example of an equation, it is numbered within the chapter. 

% Equations are numbered within chapters. Typeset them the way you normally do
\begin{equation}
\pi\times r^2
\label{equ:area}
\end{equation}

\section{second section!}

\subsection{subsection}

\subsubsection{subsubsection}

% I don't think most people will use paragraph and subparagraph. But they are
% included here for completeness
\paragraph{paragraph}

\subparagraph{subparagraph}

\lipsum{}
\chapter{Everyone knows, the second chapter is the best chapter}
\lipsum{}

% Also include figures the normal way, using the graphicx package, or however
% you choose to do it
\begin{figure}
\centering
\includegraphics{kgcoelogohoriz}
\caption{Trust me, this is a figure}
\label{fig:samp}
\end{figure}

% A subfigure can be created using the environments provided in the subcaption
% package
\begin{figure}
\centering
\begin{subfigure}{0.25\linewidth}
\centering
\includegraphics[width=\linewidth]{kgcoelogovert}
\caption{Left subfigure}
\label{fig:sub:left}
\end{subfigure}
\quad\quad\quad\quad
\begin{subfigure}{0.25\linewidth}
\centering
\includegraphics[width=\linewidth]{kgcoelogovert}
\caption{Right subfigure}
\label{fig:sub:right}
\end{subfigure}
\caption{This is the caption for the entire figure}
\label{fig:sub}
\end{figure}

some more\footnote{aoeu} text

% Tables are unchanged as well
\begin{table}
\centering
\caption{Example of a table}
\label{tab:samp}
\begin{tabular}{cc}
\toprule
a		& b\\
\midrule
1		& 5\\
2		& 9\\
3		& 2\\
4		& 2\\
\bottomrule
\end{tabular}
\end{table}

\chapter[Example chapter]{Example chapter with an excessively long title such that it might be forced to break. Apparently it works}
\lipsum[8-12]

% Citing some stuff
\nocite{cheung,zhao,cao,zhang}

% Citing everything
\nocite{*}

% Making the bibliography, using the IEEE format. By default the IEEE format
% won't replace author's names with et al., even when there are a lot of
% authors. Consult the IEEE LaTeX format documentation for how to change that
\makebibliography{example}

% The theappendix environment is used when there is only one appendix.
% Otherwise use \appendix. Do not use both the theappendix environment
% and \appendix
\begin{theappendix}{Single appendix title}
	\section{stuff}
	\lipsum[15-21]
\end{theappendix}

% \appendix begins the appendicies. If you have just one appendix, it should
% technically be just "Appendix" not "Appendix A" and should be referred to in
% the text as "the appendix".
\appendix

% Begin a new appendix by starting new chapters
\chapter{Sample Appendix}

% You can make sections in appendicies, but I don't think many people will
\section{test section---probably not necessary}
\lipsum[6-12]
\chapter{Another Sample Appendix}
\lipsum[10-11]

% and finally, end the document
\end{document}
