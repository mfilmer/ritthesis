% This class inherits from the LaTeX standard report class and should support
% all of the options supported by it, as well as a few extra. These extra
% options are:
% singlespacing, onehalfspacing, doublespacing
% Only one should be used at a time, and the default is doublespacing
\documentclass{ritthesis}

%%%%%%%%%%%%%%%%%%%%
%%%%% Packages %%%%%
%%%%%%%%%%%%%%%%%%%%
% Used for filler text but will almost certainly be of no use in a real thesis
\usepackage{lipsum}
% Generally you will want to use the graphicx package to display images
\usepackage{graphicx}

% Use these commands to set up the author, title, and date
\author{Matthew J. Filmer}
% You most likely need to insert a line break or two to make the title wrap
% more evenly
\title{A \LaTeX{} Document Class for a Masters Thesis\\in Microelectronic Engineering at RIT}
\date{June 23, 2014}

% You can further customize the apperance of the title and author by
% uncommenting and changing:
% \makeatletter
% \renewcommand{\displaytitle}{\textbf{\large\@title}}
% \renewcommand{\displayauthor}{\textsc{\@author}}
% \makeatother

% There are also a bunch of other commands that can be changed, were this not
% a masters thesis in microelectronic engineering at RIT. Those would be the
% following:
% \renewcommand{\universityname}{Rochester Institute of Technology}
% \renewcommand{\collegename}{Kate Gleason College of Engineering}
% \renewcommand{\departmentname}{Department of Electrical and Microelectronic Engineering}
% \renewcommand{\logo}{kgcoelogo}
% \renewcommand{\degreename}{Master of Science}
% \renewcommand{\degreefield}{Microelectronic Engineering}

% Each of these commands can be repeated as many times as necessary
% Used to create the lines on the signature sheet
\advisor{Dr. Joe Smith}
\committee{Dr. Philip Jones}
% Note, the following line makes the signature area overflow and should
% generally be shortened. This is probably an extreme case anyway.
\specialcommittee{Bystander}{Mr. Robert van Rosenson XXXVIII}{King of the Committee}

% You can customize the length of the signature line by uncommenting and
% changing the following line. The default is 20em, and is close to the maximum
% line length
%\renewcommand{\siglinelength}{20em}

% Begin the document, the way you always do in LaTeX
\begin{document}

% This starts the pages that get roman numerals at the beginning of the
% document. Generally is the first thing in the document. 
\makefrontmatter

% Instead of \makefrontmatter it is possible to do what thath macro does
% manually, if more control was desired. If that is the case, uncomment and
% change, or just use the individual macros:
% \renewcommand{\makefrontmatter}{\maketitlepage\makesignaturepage\makecopyrightpage}
% Each of these commands can also be changed, but for details on that you
% should look in ritthesis.cls to see how they are defined

% The acknowledgments environment will create an acknowledgments section,
% complete with a correctly spelled title
\begin{acknowledgments}
\lipsum[4-6]
\end{acknowledgments}

% The abstract environment is very similar to the acknowledgments environment
% and is used the same
\begin{abstract}
\lipsum[1-2] 
\end{abstract}

% Makes the table of contents, list of tables, and list of figures, in that
% order. This command makes a table of contents entry for each of these lists
\makealllists

% The \mainmatter macro starts the main matter. This starts arabic numbering,
% and adds a header (except on the first page of a chapter). 
\mainmatter

% This class uses the fancyhdr package to build the headers. You could probably
% customize the headers by changing the plain page style, as well as the fancy
% page style. There is no guarantee anything you do works though. (In fact,
% there isn't even a guarantee anything in ritthesis.cls works either)

% Likewise, the chapter and section headers are formatted using the titlesec
% package. You should be able to customize them by using that package's
% commands, specifically \titleformat

% Begin new chapters the way you always do it LaTeX
\chapter{Doing things the hard way}

% Same goes for sections
\section{First sections aren't as much fun as first comments}
The first paragraph in a chapter\footnote{and probably section} is not indented. I can probably change this, but it is the \LaTeX{} default. Here is an example of an equation, it is numbered within the chapter. 

% Equations are numbered within chapters. Change the way you normally do
\begin{equation}
\pi\times r^2
\label{equ:area}
\end{equation}

\section{second section!}

\subsection{subsection}

% I don't think most people will use paragraph and subparagraph. But they are
% included here for completeness
\paragraph{paragraph}

\subparagraph{subparagraph}

\lipsum{}
\chapter{Everyone knows, the second chapter is the best chapter}
\lipsum{}

% Also include figures the normal way, using the graphicx package, or however
% you choose to do it
\begin{figure}
\centering
\includegraphics{kgcoelogo}
\caption{trust me, this is a figure}
\label{fig:samp}
\end{figure}

some more\footnote{aoeu} text

% Tables are unchanged as well
\begin{table}
\centering
\caption{example of a table}
\label{tab:samp}
\begin{tabular}{cc}
\hline
a		& b\\
\hline
1		& 5\\
2		& 9\\
3		& 2\\
4		& 2\\
\hline
\end{tabular}
\end{table}

% Citing some stuff
\nocite{cheung,zhao,cao,zhang}

% Citing everything
\nocite{*}

% Making the bibliography, using the IEEE format. By default the IEEE format
% won't replace author's names with et al., even when there are a lot of
% authors. Consult the IEEE LaTeX format documentation for how to change that
\bibliographystyle{IEEEtran}
\bibliography{example}

% \appendix begins the appendicies. If you have just one appendix, it should
% technically be just "Appendix" not "Appendix A" and should be referred to in
% the text as "the appendix". I think there might be an appendix environment
% as a part of the report class for if you just have one appendix. Consult the
% internet on this one.
\appendix

% Begin a new appendix by starting new chapters
\chapter{Sample Appendix}

% You can make sections in appendicies, but I don't think many people will
\section{test section -- probably not necessary}
\lipsum[6-9]
\chapter{Another Sample Appendix}
\lipsum[10-11]

% and finally, end the document
\end{document}